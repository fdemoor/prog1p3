\documentclass[a4paper]{article}

\usepackage[english]{babel}
\usepackage[utf8]{inputenc}
\usepackage[T1]{fontenc} 
\usepackage[margin=2.5cm]{geometry}
\usepackage{graphicx}
\usepackage{xspace}

\usepackage{amsmath}
\usepackage{amsfonts}
\usepackage{amssymb}

%% Tikz for picture drawing %%
\usepackage{tikz}
\usetikzlibrary{arrows}

\usepackage{xcolor,colortbl}

\usepackage{caption}
\usepackage{subcaption}
\usepackage{float}

\begin{document}

\title{PROG1 P3 - Ant vs. SomeBees}
\author{Simon Bihel, Florestan De Moor \\ ENS Rennes, Computer Science Department, 1st year}
\date{December 13rd, 2015}

\newcommand{\this}{\emph{Ant vs. SomeBees }}

\maketitle

\section*{Introduction}


\section{Game overview}
%% TODO Enough, or more precise ? %%

\paragraph{} \this is a tower defence like game. The terms are really simple : the player controls an anthill, which has five tunnels. An army of bees will try to invade it. The goal is to survive as long as possible by killing the bees entering the tunnels in waves. If a bee succeed to go through a whole tunnel, the game is lost. Each bee killed gives the player 5 points, the player score is printed in the bottom information bar.

\paragraph{Defending the tunnels} In order to defend the tunnels, the player has the possibility to put some ants. This has a cost : food. The starting food amount is 2. The food amount is printed in the bottom bar. Each ant has specific cost, armour, and abilities. For example, the harvester ant collect 1 food per turn, while thrower ants can throw projectiles to attack bees. The bodyguard is a special unit, that can be put at the same place than another ant, in order to protect it.

\paragraph{Bee waves} The first bee will come in after six turns, and one bee will appear at each turn from then on. The wave difficulty will increase : incoming bees will have greater armour, and damages up. There are two types of bees : the black ones, that can attack ants when they are right in front of them, and the red ones, that can shoot afar. A new bee appears in one of the five tunnels, picked up randomly.

\paragraph{Spells} The player has also the possibility to use spells. This also costs food. Three spells were implemented :

\begin{itemize}
	\item The freeze ability can be applied on a place of the tunnel, and has for consequence to freeze all bees within for three turns. Notice that they are unable to move, but not to shoot.
	\item The radar ability can be given to an ant. As long as the concerned ant is living, the player will be able to see the armour points of all bees in the specific tunnel.
	\item The upgrade spell is very powerful : it can be applied to an ant, in order to level it up. This doubles its damage power. Notice that it doesn't modify armour points, the player has to protect well its upgraded ants.
\end{itemize}


\section{Class diagrams}

	\subsection{Game engine}
	\subsection{Graphic user interface}


\section{Implementation clarifications}


\section{Possible improvements}


\section*{Conclusion}



\end{document}